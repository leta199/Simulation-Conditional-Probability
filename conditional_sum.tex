
\documentclass{article}
\usepackage{amsmath, amssymb}
\usepackage{geometry}
\geometry{margin=1in}

\begin{document}

When you roll two dice, we define our sample space as the result of the first die roll and the sum of both dice.

\[
\Omega = \left\{ \text{(first dice roll, sum of dice)} \right\}
\]

Define event \( A \) as:

\[
A = \left\{ \text{first die roll is 2, sum of dice is 7} \right\}
\]

Let \( X_1 \) and \( X_2 \) be random variables representing the outcomes of the first and second dice, respectively.  
The sum is represented as \( X_1 + X_2 \).

We are interested in computing the conditional probability,

\[
P(A) = P(X = 2 \mid X_1 + X_2 = 7)
\]

Using the definition of conditional probability,

\[
P(X = 2 \mid X_1 + X_2 = 7) = \frac{P(X = 2 \cap X_1 + X_2 = 7)}{P(X_1 + X_2 = 7)}
\]

We first find all outcomes such that \( X_1 + X_2 = 7 \):

\[
\{ (1,6), (2,5), (3,4), (4,3), (5,2), (6,1) \}
\]

Of these six outcomes, only one has \(X_1 = 2 \), namely \( (2,5) \).

So:

\[
P(X = 2 \mid X_1 + X_2 = 7) = \frac{1}{6}
\]

\end{document}
